\documentclass[11pt]{article}
\usepackage{url}
\usepackage{alltt}
\usepackage{bm}
\linespread{1}
\textwidth 6.5in
\oddsidemargin 0.in
\addtolength{\topmargin}{-1in}
\addtolength{\textheight}{2in}

\usepackage{amsmath}
\usepackage{amssymb}
\usepackage{hyperref}

\begin{document}


\begin{center}
\Large
Class activity: Reading a research paper\\
\normalsize
\vspace{5mm}
\end{center}



\section*{Reading a research paper}

Statistics is an important research tool used in many fields. In this activity, you will read the original research article analyzing the dengue data which we have used in class. The purpose of this assignment is to help you learn how to read a research paper, and extract key details about the study and results. Later, we will try to reproduce the authors' results ourselves.

\subsection*{Overview}

Dengue is a mosquito-borne viral disease which affects hundreds of millions of people each year. Early diagnosis is crucial for patients to have the best prognosis, but relies on a variety of laboratory tests. To enhance practitioners' ability to diagnose dengue, a 2015 paper by Tuan \textit{et al.}\footnote{Tuan, Nguyen Minh, et al. (2015) ``Sensitivity and specificity of a novel classifier for the early diagnosis of dengue.'' \textit{PLoS neglected tropical diseases} 9.4 (2015): e0003638.}  investigated the possibility of detecting dengue using a variety of clinical measurements like white blood cell count and platelet count. This activity will help you read the paper by Tuan \textit{et al.}. The paper is available at

\url{https://journals.plos.org/plosntds/article?id=10.1371/journal.pntd.0003638}

\subsection*{Outline of a research paper}

Research papers in many fields, particularly the sciences, often contain the following main sections:

\begin{itemize}
\item  \textbf{Abstract:} A short overview of the full paper, giving highlights of the motivation and background, the research question, the data, and the results.
\item \textbf{Introduction:} A broad overview of the research question the authors want to study, motivation for studying this question, and the authors' approach to answering their question. The introduction often starts very general, then narrows to the specific question addressed in this paper. More detail is provided in the introduction than in the abstract, and more time is spent on motivation and related literature.
\item \textbf{Methods:} The data and analysis techniques used to answer the research question. This typically describes the what the data looks like, how and where it was collected, and any statistical tools (e.g. visualizations, regression, hypothesis testing) that were used when analyzing the data.
\item \textbf{Results:} A summary of the analysis results, such as figures showing regression fits, and tables of regression coefficients and p-values.
\item \textbf{Discussion:} A discussion of the analysis results, in context of the original research question. In this section, explanations for \textit{why} particular results were observed may be proposed.
\item \textbf{Conclusion:} A short summary of the paper and its key results, and their connections to broader scientific questions. The conclusion is often the reverse of the introduction: it starts with the specific question addressed by this paper, then discusses the implications of this research for science in general.
\end{itemize}



\subsection*{Reading a research paper}

Reading a research paper, particularly in a field in which you are not an expert, can be challenging. The trick is to skim the paper for the most relevant information, and skip over technical details that are not essential to understanding the key take-aways. The questions below will guide you to the most important sections in the paper by Tuan \textit{et al}.


\subsection*{The Abstract and Introduction}

A good place to start is often with the Abstract and Introduction, which allow you get an overview of the paper, and usually don't contain too many technical details. The Abstract is more succinct than the Introduction, but it also provides less motivation. When the Introduction is long, you may want to skim for key details.\\

\noindent Read the Abstract and introduction. Then answer the following questions.

\begin{enumerate}
    \item Why is it important for the researchers to build a model to detect dengue in hospital patients?
    
    \item What is the specific purpose of the research study?
\end{enumerate}

\noindent So far, we know what question the researchers are trying to answer, and we know that they are going to build some kind of model to predict dengue. Our goal for the rest of the paper is to understand how the authors conducted this analysis. In particular, we want to answer the following questions:

\begin{itemize}
\item Who participated in the study?
\item What did the researchers record about each participant?
\item What statistical methods did the researchers use?
\item What did the researchers conclude from their study?
\end{itemize}

\noindent This information is provided in the Methods and Results sections of the paper. These sections also contain lots of other details which is valuable, but not crucial to understand on a first reading, so we will focus on the most important parts of the Methods and Results.

\subsection*{Study Participants}

Read the \textit{Patient Enrolment} subsection of the Methods, and then answer the following questions.

\begin{enumerate}
    \item[3.]  How many patients participated in the study?
\end{enumerate}

\noindent Potential research subjects must meet certain criteria, defined by the researchers, to participate in a study. \textit{Inclusion} criteria define requirements for inclusion in the study (e.g., a target age range or social group), while \textit{exclusion} criteria are reasons a subject would be asked not to participate (e.g., certain medical conditions). 

\begin{enumerate}
    \item[4.] What are the inclusion/exclusion criteria for this study?
\end{enumerate}


\subsection*{Data Collection and Analysis}

Now that we know who was studied, we want to know what data was collected about each participant, and how it was analyzed. Read the \textit{Clinical and laboratory investigations on the day of enrolment} and \textit{Statistical methods} subsections of the Methods. Then answer the following questions.

\begin{enumerate}
\item[5.] Which variables were recorded for the patients in the study?

\item[6.] Which types of statistical methods were used to model the relation between the explanatory variables and whether the patient had dengue? (It is ok if you're not familiar with all these methods!)

\item[7.] How did the researchers choose which variables to include in their logistic regression model?

\item[8.] Which threshold did the researchers use when converting their predicted probabilities into binary predictions?

\end{enumerate}


\subsection*{Results}

Finally, let's see what the researchers concluded from their statistical models. Read the \textit{Early Dengue Classifier} subsection of the Results, then answer the following questions.

\begin{enumerate}
    \item[9.] Which variables did the researchers include in their final logistic regression model?
    
    \item[10.] What metrics did the researchers use to assess predictive ability of their model?
    
    \item[11.] What sensitivity and specificity values did the researchers observe using their final model?
    
    \item[12.] What Area Under the ROC Curve (AUC) did the researchers observe?
\end{enumerate}


\end{document}
