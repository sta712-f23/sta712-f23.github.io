\documentclass[11pt]{article}
\usepackage{url}
\usepackage{alltt}
\usepackage{bm}
\linespread{1}
\textwidth 6.5in
\oddsidemargin 0.in
\addtolength{\topmargin}{-1in}
\addtolength{\textheight}{2in}

\usepackage{amsmath}
\usepackage{amssymb}
\usepackage{bbm}
\usepackage{bm}
\usepackage{hyperref}


\title{STA 712: HW 2}

\author{}
\date{}

\begin{document}

\maketitle

\noindent \textbf{Due:} Friday, September 15, 12:00pm (noon) on Canvas\\

\noindent \textbf{Instructions:} Submit your work as a single PDF. Your document should be created using LaTeX; see the course website for a \href{https://sta712-f23.github.io/homework/hw_template.tex}{homework template file} and \href{https://sta712-f23.github.io/homework/latex_instructions/}{instructions} on getting started with LaTeX and Overleaf.

\section{Reproducing the dengue results}

Now that you've read the dengue paper by Tuan \textit{et al.}, we will try to reproduce their results. I have downloaded their data, and performed some initial data cleaning for you. The prepared data can be loaded into R using the following command:

\begin{verbatim}
dengue <- read.csv("https://sta214-s23.github.io/homework/dengue.csv")
\end{verbatim}

The prepared data contains 5720 patients, with the following variables:

\begin{itemize}
\item SiteNumber: The hospital at which the data was recorded 
\item Sex: patient's sex (female or male)
\item Age: patient's age (in years)
\item DiseaseDay: how long the patient has been ill
\item Vomiting: whether the patient has experienced vomiting (0 = no, 1 = yes)
\item Abdominal: whether the patient has abdominal pain (0 = no, 1 = yes)
\item Temperature: patient's body temperature (in Celsius)
\item BMI: the patient's body mass index (BMI)
\item WBC: the patient's white blood cell count
\item HCT: the patient's hematocrit
\item PLT: the patient's platelet count
\item RapidTest: predicted disease status from a rapid test (positive or negative)
\item Dengue: whether the patient actually has dengue fever, based on a lab test (0 = no, 1 = yes)
\end{itemize}

\begin{enumerate}
\item[12.] First, let's look at the rapid test. 

\begin{enumerate}
\item Create a confusion matrix for the predictions from the rapid test. Note that you will not need to threshold these predictions, as the rapid test already makes binary predictions!

\item Calculate the accuracy, sensitivity, specificity, and positive predictive value for the rapid test.
\end{enumerate}

\item[13.] Next, let's look at the final model chosen by the researchers. Their Early Dengue Classifier uses age, white blood cell count, and platelet count to predict dengue status. 

\begin{enumerate}
\item Fit a logistic regression model to predict dengue status using age, white blood cell count, and platelet count.

\item Create a confusion matrix for the predictions from your fitted model. Use the same threshold as in the paper (0.333).

\item Calculate the accuracy, sensitivity, specificity, and positive predictive value for your logistic regression model. Are the values close to the values reported in the original paper? (It is ok if they don't match exactly)

\item How does the logistic regression model perform, compared to the rapid test?

\item Now let's create an ROC curve for the logistic regression model, so we can assess predictive performance across different thresholds. If your logistic regression model is named \verb;m1;, the following code will create the ROC curve and calculate the AUC. Run the code below to calculate the AUC and make the plot; is the AUC similar to the value reported in the original paper?

\begin{verbatim}
library(ROCR)
pred <- prediction(m1$fitted.values, dengue$Dengue)
perf <- performance(pred,"tpr","fpr")

performance(pred, "auc")@y.values 

data.frame(fpr = perf@x.values[[1]],
           tpr = perf@y.values[[1]]) %>%
  ggplot(aes(x = fpr, y = tpr)) +
  geom_line(lwd=1.5) +
  geom_abline(slope = 1, intercept = 0, lty = 2,
              lwd = 1.5) +
  labs(x = "False positive rate (1 - Specificity)",
       y = "True positive rate (Sensitivity)") +
  theme_classic()
\end{verbatim}

\end{enumerate}

\item[14.] Finally, let's experiment with model selection to see if we get a different model than the one selected by the researchers.

\begin{enumerate}
\item Adapt the code from the class activity on February 13 to perform forward stepwise selection with AIC on the dengue data. Your response variable should be \verb;Dengue;, and your full model (the \verb;scope; in the \verb;stepAIC; function) should contain all explanatory variables \textbf{except} for RapidTest and SiteNumber. Which variables are chosen in forward stepwise selection?

\item Calculate an AUC for the model chosen by forward stepwise selection. Is it very different from the AUC of the model in question 13?

\item Explain why the researchers preferred the model from question 13.
\end{enumerate}
\end{enumerate}

\section{Data analysis}

You are contacted by the US Small Business Administration (SBA), a government agency dedicated to helping support small businesses. The SBA provides loans to small businesses, but some businesses \textit{default} on their loan (i.e., fail to pay it back). Researchers at the SBA are interested in predicting whether a business will default on the loan, and they have collected a random sample of 5000 different loans.\\

\noindent You can load the SBA data into R by

\begin{verbatim}
sba <- read.csv("https://sta712-f22.github.io/homework/sba_small.csv")
\end{verbatim}


\noindent For each loan, we have the following variables:

\begin{itemize}
\item LoanNr\_ChkDgt:	Loan ID number that uniquely identifies each loan
\item Name: 	Name of business receiving the loan
\item City: 	City the business is based in
\item State: 	State the business is based in (two-letter abbreviation)
\item Zip: 	ZIP code the business is based in
\item Bank: 	Name of bank making the loan
\item BankState: 	State of the bank making the loan (two-letter abbreviation)
\item NAICS: 	North American Industry Classification System code identifying the industry of the business receiving the loan
\item ApprovalDate: 	Date of approval (YYYY-MM-DD) of the loan
\item ApprovalFY: 	Fiscal year of approval of the loan
\item Term: 	Length of the loan term (months)
\item NoEmp: 	Number of employees of the business before receiving the loan
\item NewExist: 	1 if business already existed, 2 if business is new
\item CreateJob: 	Number of jobs the business expects to create using the loan money
\item RetainedJob: 	Number of jobs the business expects to retain because they received the loan
\item FranchiseCode: 	For businesses that are franchises, a unique five-digit code identifying which brand they are a franchise of. 0 or 1 if the business is not a franchise.
\item UrbanRural: 	1 if business is in urban area, 2 if business is in rural area, 0 if unknown
\item RevLineCr: 	Y if this is a revolving line of credit, N if not
\item LowDoc: 	Y if loan was issued under the `LowDoc Loan' program, which allows loans under \$150,000 to be processed with a short one-page application. N if loan is issued with a standard application, which is much longer
\item ChgOffDate: 	The date (YYYY-MM-DD) the loan was declared to be in default, if the borrower stopped paying it back
\item DisbursementDate: 	Date (YYYY-MM-DD) the loan money was disbursed to the business
\item DisbursementGross:	The amount of money disbursed (loaned), in dollars
\item BalanceGross: 	The amount of money remaining to be paid back, in dollars
\item MIS\_Status: 	Current loan status. CHGOFF = charged off, P I F = paid in full.
\item ChgOffPrinGr: 	Amount of money charged off, if the borrower defaulted, in dollars
\item GrAppv: 	Gross amount of loan approved by the bank, in dollars
\item SBA\_Appv: 	Amount of the loan guaranteed by the SBA, in dollars 
\end{itemize}

\noindent \textbf{Research question:} Suppose that researchers at the SBA would like to predict which loans are likely to default; this will help them decide how to best allocate loans to the businesses applying. They ask you to build a model to predict loan default, taking into account the following preferences:
\begin{itemize}
\item Saving money is a top priority, so they want a model that will do a good job at predicting defaults for new loans

\item However, the final model also has to be understandable by loan officers and the businesses applying. In particular, the SBA wants to be able to explain to businesses \textit{why} an application has been denied.
\end{itemize}

\noindent Taking these preferences into account, build a model to predict loan defaults using the SBA data. Then answer the following questions. Make sure to provide all code and results. (Note: the questions do not ask you about exploratory data analysis, but EDA is always a good idea).

\begin{enumerate}
\item Summarize your final model, and describe how you chose that model. 

\item Interpret some of the coefficients in your final model.

\item Assess and interpret the predictive performance of your final model.
\end{enumerate}

\section{Two views of AIC}

\subsection*{AIC and the KL divergence}

Let $Y$ denote the response variable of interest in a generalized linear model, and let $f_0(y)$ denote the \textit{true} probability function of the response (i.e., the true distribution that generated the data). For simplicity, assume here that $Y$ is continuous, so $f_0(y)$ is a density.\\

\noindent We consider a family of distributions $\{f_{\theta}(y) : \bm{\theta} \in \Theta \}$, parameterized by a vector $\bm{\theta} \in \mathbb{R}^p$. Our goal, when building a model, is to choose the parameters $\bm{\theta}$ which make $f_\theta(y)$ as close to the \textit{true} distribution $f_0(y)$ as possible.\\

\noindent One method of measuring how close $f_\theta(y)$ is to $f_0(y)$ is with the \textbf{Kullback-Leibler} (KL) divergence:
\begin{align}
K(f_\theta, f_0) = \int \log \left( \frac{f_0(y)}{f_\theta(y)} \right) f_0(y) dy.
\end{align}
We fit our model via maximum likelihood estimation, producing the maximum likelihood estimate $\widehat{\bm{\theta}}$ (which depends on the observed data). Then, $K(f_{\widehat{\theta}}, f_0)$ is one measure of how close our model is expected to fit a \textit{new} set of data. And one way of choosing a model would be to choose the model which does best at predicting new data, i.e. which minimizes the KL divergence.\\

\noindent In the following questions, you will explore some basics of KL divergence, and show that choosing a model to minimize AIC is equivalent to minimizing an estimate of the KL divergence.

\begin{enumerate}
\item Show that $K(f_\theta, f_0) = 0$ if $f_\theta = f_0$.

\item Using the fact that $\log(x) \leq x - 1$ for all $x$, show that $K(f_\theta, f_0) \geq 0$ if $f_\theta \neq f_0$.

\item Conclude that, if we \textit{could} calculate KL divergence for our models, and the true model $f_0$ was among the models considered in the model selection procedure, then minimizing KL divergence would find the true model.

\item Explain why we can't actually calculate KL divergence when modeling real data.
\end{enumerate}

\noindent Since we can't calculate KL divergence, we to approximate it. Let $\bm{\theta}^*$ be the value of $\bm{\theta}$ which minimizes $K(f_\theta, f_0)$. A Taylor expansion gives
\begin{align}
K(f_{\widehat{\theta}}, f_0) \approx K(f_{\theta^*}, f_0) + \frac{1}{2} (\widehat{\bm{\theta}} - \bm{\theta}^*)^T \mathcal{I}(\bm{\theta}^*) (\widehat{\bm{\theta}} - \bm{\theta}^*),
\end{align}
where $\mathcal{I}(\bm{\theta}^*)$ is the information matrix at $\bm{\theta}^*$.

\begin{enumerate}
\item We know from STA 711 that, if the model is correct and the sample size is sufficiently large, then $(\widehat{\bm{\theta}} - \bm{\theta}^*)^T \mathcal{I}(\bm{\theta}^*) (\widehat{\bm{\theta}} - \bm{\theta}^*) \approx \chi^2_p$. Conclude that under these assumptions,
\begin{align}
\mathbb{E}[K(f_{\widehat{\theta}}, f_0)] \approx K(f_{\theta^*}, f_0) + p/2.
\end{align}
\end{enumerate}

\noindent So far, our approximation to $K(f_{\widehat{\theta}}, f_0)$ is $K(f_{\theta^*}, f_0) + p/2$ (we can already see a penalty term that involves the number of parameters!). However, $K(f_{\theta^*}, f_0)$ still depends on $f_0$. To estimate $K(f_{\theta^*}, f_0)$, let $\bm{Y} = [Y_1,...,Y_n]^T$ denote the observed data used to calculate $\widehat{\bm{\theta}}$, and consider the log-likelihood $\ell(\widehat{\bm{\theta}}) = -\sum_i \log f_{\widehat{\theta}}(Y_i)$.

\begin{enumerate}
\item Under the same assumptions as the previous question, $2(\ell(\widehat{\bm{\theta}}) - \ell(\bm{\theta}^*)) \approx \chi^2_p$. Use this to argue that 
\begin{align}
K(f_{\theta^*}, f_0) \approx \mathbb{E}[- \ell(\widehat{\bm{\theta}})] + p/2 + \int \log(f_0(y)) f_0(y) dy.
\end{align}

\item Combine the previous questions to argue that an estimate of $K(f_{\widehat{\theta}}, f_0)$ is
\begin{align}
\widehat{K(f_{\widehat{\theta}}, f_0)} = -\ell(\widehat{\bm{\theta}}) + p + \int \log(f_0(y)) f_0(y) dy.
\end{align}

\item Conclude that minimizing $\widehat{K(f_{\widehat{\theta}}, f_0)}$ is equivalent to minimizing
\begin{align*}
AIC = -2\ell(\widehat{\bm{\theta}}) + 2p.
\end{align*}
\end{enumerate}

\noindent This derivation isn't super rigorous, but it captures the main steps and the important intuition: AIC is a reasonable metric for model selection because minimizing AIC is equivalent to minimizing an estimate of the KL divergence.

\subsection*{AIC and LOOCV}

Another way of viewing AIC is as a computationally efficient approximation of leave-one-out cross-validation (LOOCV). Indeed, Stone (1977) showed that (under appropriate assumptions), \textit{AIC and LOOCV are asymptotically equivalent}. Stone's paper is quite short (indeed, it is actually published in the journals \textit{Notes, Queries, and Comments} rather than as a full-length research article), and this part of the assignment will guide you through reading the original paper.\\

\begin{quote}
Stone, M. (1977). \href{https://www.jstor.org/stable/2984877}{An asymptotic equivalence of choice of model by cross‐validation and Akaike's criterion}. \textit{Journal of the Royal Statistical Society: Series B (Methodological)}, 39(1), 44-47.
\end{quote}

\noindent Begin by reading sections 1 -- 3 of the paper, which set up the background and problem the author is trying to solve. (Because this is a short research note, rather than a full paper, the structure of the article is a bit different than usual).

\begin{enumerate}
\item In this paper, what is the LOOCV metric that we want to minimize? That is, what value is calculated for each held-out observation?

\item The goal of the article is to show that model selection by AIC and LOOCV are asymptotically equivalent. In particular, which two quantities (describe the quantities and give the equation numbers) does the author plan to show are equivalent?
\end{enumerate}

\noindent Now read section 4 (you may need to read it several times). For your first read-through, I suggest skimming over details like regularity conditions, and focusing on the general outline of the mathematical results. In particular:

\begin{enumerate}
\item What technique (which we have used many times in class!) does the author use to characterize the large-sample behavior of $A$?

\item What is the key assumption that makes $A$ asymptotically equivalent to AIC?
\end{enumerate}

\noindent Finally, read through section 4 again, keeping the outline and direction of the derivation in mind as you read. This time, pay more attention to the mathematical details and see if you can follow the steps. The key equations are (4.4), (4.5), and (4.6).

\begin{enumerate}
\item What asymptotic results are used to move from equation (4.4) to equation (4.5)?

\item How does the ``key assumption'' allow us to write equation (4.5) as equation (4.6)? In particular, what is required for $L_1 = -L_2$? (We have seen this equivalence before in STA 711!)
\end{enumerate}

\subsection*{Reflecting}

In this assignment, we have seen two justifications of AIC. First, using AIC is reasonable because minimizing AIC is the same as minimizing an estimate of the KL divergence. And minimizing an estimate of the KL divergence is reasonable because \textit{if} the true model is contained in our search, and \textit{if} we could minimize the actual KL divergence (not the estimate) then we would choose the true model. Second, minimizing AIC is reasonable because it is asymptotically equivalent to minimizing LOOCV error. As long as we think LOOCV error is a good method of assessing model performance, then AIC should be too.\\

\noindent Let's finish the assignment by comparing these two different perspectives of the AIC.

\begin{enumerate}
\item  What are the similarities between the two perspectives of AIC (KL divergence and LOOCV)? Think about:

\begin{itemize}
\item What KL divergence and LOOCV are trying to measure
\item Similar methods used in both perspectives
\end{itemize}

Provide an intuitive explanation for why the LOOCV error in the Stone (1977) paper (the log likelihood calculated with each training observation held out in turn) would be similar to the KL divergence (which involves expected log-likelihoods...)
\end{enumerate}

\end{document}