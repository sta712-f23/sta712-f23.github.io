\documentclass[11pt]{article}
\usepackage{url}
\usepackage{alltt}
\usepackage{bm}
\linespread{1}
\textwidth 6.5in
\oddsidemargin 0.in
\addtolength{\topmargin}{-1in}
\addtolength{\textheight}{2in}

\usepackage{amsmath}
\usepackage{amssymb}
\usepackage{bm}

\begin{document}


\begin{center}
\Large
STA 712 Challenge Assignment 3: Logistic regression in Python\\
\normalsize
\vspace{5mm}
\end{center}

\noindent \textbf{Due:} By Friday, November 10 at 12:00pm (noon) on Canvas.\\

\noindent \textbf{Instructions:} 
\begin{itemize}
\item Submit your work as a single typed PDF (you should not need to type much, if any, math on this assignment).
\item You are welcome to work with others on this assignment, but you must submit your own work.
\item You can probably find the answers to many of these questions online. It is ok to use online resources! And using online documentation and examples is a very important part of coding.
\end{itemize}

\section*{R vs. Python for statistics and data science}

Our language of choice in this class has been R, which is a common and popular choice for fitting and working with statistical models. R is particularly good for many core statistical tools: there is excellent support for linear models (and variants like weighted regression and robust regression), GLMs, GAMs, mixed effects models, etc. Through \texttt{tidyverse} packages like \texttt{tidyr}, \texttt{dplyr}, and \texttt{ggplot}, R is also a good choice for data cleaning, manipulation, and visualization.\\

\noindent Python is another language which is becoming increasingly popular for data science and machine learning. The \texttt{scikit-learn} module contains a wide variety of tools for fitting prediction models like regressions, support vector machines, and random forests. An advantage of \texttt{scikit-learn} is that all models have a similar structure: you can fit them using the \texttt{.fit()} function, you can get predicted probabilities with the \texttt{.predict\_proba()} function, etc.\\

\noindent Whether you use R or Python (or SAS, or SPSS, or Stata, etc.) ultimately depends on a combination of personal preferences and the task at hand. The purpose of this challenge assignment is to introduce you to fitting models in Python. Do we need Python to fit logistic regression? No -- R is pretty great at this. But it is valuable to see how Python works (and how it behaves differently to R). In the process, you will also see the general procedure for fitting a model using \texttt{scikit-learn}, and you will be briefly introduced to other important Python modules like \texttt{numpy}, \texttt{pandas}, and \texttt{scipy}.

\section*{Set up}

To complete this challenge assignment, you will need to install Python on your computer. If you do not already have Python installed (or even if you do!) I recommend installing the Anaconda distribution (\url{https://www.anaconda.com/products/distribution}). You will also need to install the following modules:
\begin{itemize}
\item \texttt{pandas}
\item \texttt{numpy}
\item \texttt{scikit-learn}
\item \texttt{scipy}
\item \texttt{matplotlib}
\item \texttt{statsmodels}
\end{itemize}
If you install Anaconda, all of these except \texttt{statsmodels} should already be included. To install \texttt{statsmodels}, see the instructions at \url{https://www.statsmodels.org/stable/install.html}.\\

\noindent Once Python is installed, how do you use it? If you have the latest versions of R and RStudio installed, you can actually use Python in RStudio! RStudio supports Quarto documents (these are one of the options when you create a new document in RStudio), which behave similarly to RMarkdown documents. In a Quarto document you can include chunks of Python code; see \url{https://quarto.org/docs/computations/python.html} to get started.

\section*{Data}

You are contacted by the US Small Business Administration (SBA), a government agency dedicated to helping support small businesses. The SBA provides loans to small businesses, but some businesses \textit{default} on their loan (i.e., fail to pay it back). Researchers at the SBA are interested in predicting whether a business will default on the loan, and they have collected a random sample of 5000 different loans.\\

\noindent You can load the SBA data into R by

\begin{verbatim}
sba <- read.csv("https://sta712-f23.github.io/homework/sba_small.csv")
\end{verbatim}


\noindent For each loan, we have the following variables:

\begin{itemize}
\item LoanNr\_ChkDgt:	Loan ID number that uniquely identifies each loan
\item Name: 	Name of business receiving the loan
\item City: 	City the business is based in
\item State: 	State the business is based in (two-letter abbreviation)
\item Zip: 	ZIP code the business is based in
\item Bank: 	Name of bank making the loan
\item BankState: 	State of the bank making the loan (two-letter abbreviation)
\item NAICS: 	North American Industry Classification System code identifying the industry of the business receiving the loan
\item ApprovalDate: 	Date of approval (YYYY-MM-DD) of the loan
\item ApprovalFY: 	Fiscal year of approval of the loan
\item Term: 	Length of the loan term (months)
\item NoEmp: 	Number of employees of the business before receiving the loan
\item NewExist: 	1 if business already existed, 2 if business is new
\item CreateJob: 	Number of jobs the business expects to create using the loan money
\item RetainedJob: 	Number of jobs the business expects to retain because they received the loan
\item FranchiseCode: 	For businesses that are franchises, a unique five-digit code identifying which brand they are a franchise of. 0 or 1 if the business is not a franchise.
\item UrbanRural: 	1 if business is in urban area, 2 if business is in rural area, 0 if unknown
\item RevLineCr: 	Y if this is a revolving line of credit, N if not
\item LowDoc: 	Y if loan was issued under the `LowDoc Loan' program, which allows loans under \$150,000 to be processed with a short one-page application. N if loan is issued with a standard application, which is much longer
\item ChgOffDate: 	The date (YYYY-MM-DD) the loan was declared to be in default, if the borrower stopped paying it back
\item DisbursementDate: 	Date (YYYY-MM-DD) the loan money was disbursed to the business
\item DisbursementGross:	The amount of money disbursed (loaned), in dollars
\item BalanceGross: 	The amount of money remaining to be paid back, in dollars
\item MIS\_Status: 	Current loan status. CHGOFF = charged off, P I F = paid in full.
\item ChgOffPrinGr: 	Amount of money charged off, if the borrower defaulted, in dollars
\item GrAppv: 	Gross amount of loan approved by the bank, in dollars
\item SBA\_Appv: 	Amount of the loan guaranteed by the SBA, in dollars 
\end{itemize}

\noindent \textbf{Research question:} Researchers at the SBA are interested in the relationship between loan amount and whether the business defaults on the loan. They believe that whether the business is new vs. an existing business, and whether it is in an urban vs. rural environment, may also be related to the chance of defaulting. The SBA gives you the data, and asks the following question:
\begin{itemize}
\item Is there a relationship between loan amount and the probability the business defaults on the loan, after accounting for whether or not the business is new, and whether it is in an urban or rural environment?
\end{itemize}

\section*{Logistic regression in Python}

In the following questions, we will work with the SBA data. \textit{Note: I have provided some scaffolded questions here to guide your analysis, but you may still need to research \emph{how} to actually do some of these steps. E.g., \emph{``how to create a new column in pandas''}.}

\begin{enumerate}

\item At the beginning of your document (e.g., in a Python chunk at the top of your Quarto file), import all the required modules.

\item Load the SBA data into Python, using the \verb;pandas.read_csv; function.

\item List the variables in the SBA data that you will use to answer the research question above.

\item Using the \verb;MIS_Status; column, create a \textit{new} column in your SBA data called \verb;Default;, which is equal to 1 if the loan was charged off (i.e., the borrower defaulted), and 0 if the loan was paid in full (the borrower did not default).

\item Create a \textit{new} column in your SBA data called \verb;Amount; which is the loan amount. 

\item In R, categorical variables automatically get converted to indicator variables when we fit a logistic regression model. This is not true in Python; part of our data pre-processing is to create the indicator variables we need. This can be done with the \texttt{sklearn.preprocessing.OneHotEncoder} class. Create a new dataset called \verb;sba_encoded; which contains your \verb;Amount; column from Question 5, and one-hot encodings of \texttt{UrbanRural} and \texttt{NewExist}. (For the purposes of this activity, we will ignore any potential interactions between the explanatory variables). \textit{Hint: you will probably want to use} \verb;drop = `first'; \textit{in your one-hot encoding!}

\item Using the \verb;sklearn.linear_model.LogisticRegression; class, the \verb;Default; column from Question 4, and the \verb;sba_encoded; data from Question 6, fit a logistic regression model and report the estimated coefficients. \textit{Hint: you will want to use} \verb;penalty = `none'; \textit{when creating the model.}

\item Do the estimated coefficients from Question 7 agree with the estimated coefficients for the same model in R? How do your estimated coefficients change when you change the \verb;solver; in your logistic regression?

\item Using the \verb;sklearn.metrics.log_loss; function, calculate the deviance for your logistic regression model in Python, and compare to the deviance reported by R.

\item Using your fitted model in Python, perform a hypothesis test to address the first research question above: Is there a relationship between loan amount and the probability the business defaults on the
loan, after accounting for whether or not the business is new, and whether it is in an urban
or rural environment?

\item As you can see from the previous questions, the \verb;scikit-learn; module is very good for building and assessing prediction models, but is less useful for doing statistical \textit{inference}. For example, we don't get a nice summary table for our model with estimated standard errors, we need to calculate deviance separately, etc.

One way to get these nice summaries in Python is with the \verb;statsmodels; module. Using the \verb;statsmodels.GLM; class, fit the same logistic regression model as above. Use the \verb;.summary(); function to report a nice table with the estimated coefficients and standard errors. \textit{Hint: make sure to add an intercept column to the} \verb;sba_encoded; \textit{data. The} \verb;statsmodels; \textit{module does not include an intercept for you.}

\item Explain why the standard errors for the \verb;NewExist; and \verb;Intercept; coefficients are so high. How would we fix that issue?

\item Finally, let's try some regression diagnostics. Python has less support for logistic regression diagnostics than R, so we will have to write functions for these diagnostics ourselves. For simplicity, we'll just focus on making a quantile residual plot.

Write a function to generate quantile residuals for your fitted model in Question 7. You should be able to adapt code from 711; use \verb;numpy.random.uniform; to sample from a uniform distribution, and \verb;scipy.stats.norm.ppf; for the inverse CDF of a standard normal.

\item Using your function from Question 13, create a quantile residual plot for \verb;Amount;. The \verb;matplotlib.pyplot.scatter; function will be useful for creating a scatterplot in Python.
\end{enumerate}

\end{document}
